\documentclass[12pt]{ctexart}
\usepackage[scale=0.8]{geometry}
\usepackage{xifthen}
\usepackage{graphicx, float}
\usepackage{amsmath}
\usepackage{enumerate}
\usepackage{multirow}
\usepackage{fancyvrb}
\usepackage{listings}
\usepackage[dvipsnames]{xcolor}
% \setenumerate[1]{itemsep=0pt,partopsep=0pt,parsep=\parskip,topsep=5pt}
% \setitemize[1]{itemsep=0pt,partopsep=0pt,parsep=\parskip,topsep=5pt}
% \setdescription{itemsep=0pt,partopsep=0pt,parsep=\parskip,topsep=5pt}
\lstset{
    language=Python,
    basicstyle={\ttfamily},
    % morekeywords={Exception,String},
    % keywordstyle={\bfseries\itshape\color{NavyBlue}},
    keywordstyle={\bfseries\color{NavyBlue}},
    % commentstyle={\sffamily\small\itshape\color{PineGreen!60!black}},
    commentstyle={\ttfamily\small\color{PineGreen!60!black}},
    % stringstyle={\rmfamily},
    frame=single,
    breaklines=true,
    backgroundcolor={\color{yellow!0!white}},
    rulecolor={\color{purple}},
    rulesepcolor={\color{orange}},
    % framesep=1em,
    % rulesep=0.3em,
    % numbers=left,
    % numbersep=2em,
    % numberstyle={\sffamily\footnotesize},
    % emph={Robot},
    % alsoletter={.},
    % emphstyle={\bfseries\color{Rhodamine}},
    columns=fullflexible
}
\newcommand{\fpicture}[3][]{\begin{figure}[H]\centering\includegraphics[width=#3\textwidth]{#2}\ifthenelse{\isempty{#1}}{}{\caption{#1}}\end{figure}}
\newcommand{\no}[1]{\paragraph[noindent]{#1}}
% \CTEXsetup[format={\Large\bfseries}]{section}
\title{编译原理作业3}
\author{付国}
\date{2019K8009922008}
\begin{document}
% \maketitle
% \tableofcontents
\begin{Verbatim}
    首选项          ctrl K
    另存为          ctrl shift S
    导出            ctrl alt shift W

    分辨率 单位长度的像素数(ppi, 像素/英寸)

    调整图片大小    ctrl T
    颜色模式 位图/位图模式/灰度模式/RGB/CMYK/LAB
    RGB: 本身发光,在黑色背景下,通过滤色,红绿蓝可以得到其他颜色
    CMYK: 在白底下,青、品红、黄可以通过正片叠底得到其他颜色

    颜色填充 HSB(色相/饱和度/明度)
    填充前景色 alt del
    填充背景色 ctrl del
    重置前景色和背景色 D
    对调前景色和背景色 X
    
    选区
    加选 shift
    减选 alt
    交叉 alt shift
    取消 ctrl D
    重选 ctrl shift D
    反选 ctrl shift I

    去水印1: 选择大的区域,填充.内容识别
    去水印2: 选择大的区域,通过颜色精选,扩展,内容识别填充

    填充 shift F5
    选区羽化 shift F6
    新建图层,复制选区 ctrl J

    颜色矫正
    校样颜色 ctrl Y
    色域警告 ctrl shift Y

    放缩 ctrl space
    抓手 space
    按屏缩放 ctrl 0
    按实缩放 ctrl 1
    面板组显隐 tab
    屏幕模式切换 F
\end{Verbatim}
\end{document} 